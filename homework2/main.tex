\documentclass[12pt,letterpaper]{article}
\newcommand{\sethead}[5]{
	\lhead{\textit{#1}}
	\rhead{#2\\#3\\#4\\#5}
}
\usepackage[utf8]{inputenc}
\usepackage{fancyhdr}
\usepackage{setspace}
\usepackage{geometry}
\newgeometry{top=1in,bottom=1.5in,left=1in,right=1in}
\usepackage{graphicx}
\pagestyle{fancy}

\begin{document}
    \newlength{\saveparindent}
	\setlength{\saveparindent}{\parindent}
	\raggedright
	\setlength{\parindent}{\saveparindent}
	\sethead{Homework 2}{Adrian Lu and Edward Seim}{CMSI 282}{Dr. Ray Toal}{March 1, 2015}
    
	\doublespacing
	\begin{center}
	\end{center}
    
    \begin{enumerate}
        \item
            \begin{enumerate}
                \item f= \(\theta\)(g)
                \item f= \(\Omega\)(g)
                \item f= \(\Omega\)(g)
                \item f= 
                \item f= 
                \item f= \(\Omega\)(g)
                \item f= \(O\)(g)
                \item f= \(\Omega\)(g)
                \item f= \(\Omega\)(g)
                \item f= \(O\)(g)
                \item f= \(\Omega\)(g)
                \item f= \(\Omega\)(g)
                \item f= \(\Omega\)(g)
                \item f= \(\Theta\)(g)
                \item f= \(O\)(g)
                \item f= \(\Omega\)(g)
                \item f= \(\Omega\)(g)
            \end{enumerate}
        \item
            \begin{enumerate}
                \item
                    $\begin{array}{cc}
                    AW + BY & AX + BZ \\
                    CW + DY & CX + DZ \\
                    \end{array}$
                    =
                    $\begin{array}{cc}
                    A & B \\
                    C & D \\
                    \end{array}$
                    $\times$
                    $\begin{array}{cc}
                    W & X \\
                    Y & Z \\
                    \end{array}$
                \item
                    Take the example of calculating $X^8$. By separating $X^8$ into its expanded multiplication form, we get X $\cdot$ X $\cdot$ X $\cdot$ X $\cdot$ X $\cdot$ X $\cdot$ X $\cdot$ X. By taking the product of each X $\cdot$ X with $X^2$, you get $X^2$ $\cdot$ $X^2$ $\cdot$ $X^2$ $\cdot$ $X^2$. Now, by again taking the product of each $X^2$  $\cdot$ $X^2$, you get $X^4$ $\cdot$ $X^4$. Finally, by combining these terms, we arrive at $X^8$, through 3 matrix multiplications. 3 = $log_2$(8). So, to compute $X^n$, it will take \(O\)(log(n)) mathematical operations.
            \end{enumerate}
    \end{enumerate}
    
\end{document}